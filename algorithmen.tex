\documentclass[12pt]{article}
\usepackage{listings}
\usepackage{fontspec,xltxtra}
\setmainfont{Liberation Sans}
\usepackage[a4paper,left=3cm,right=3cm,top=1.5cm,bottom=2.5cm]{geometry}
\usepackage{qtree}
\usepackage[ngerman]{babel}
\usepackage{datetime}
\newdateformat{myformat}{\THEDAY \monthname[\THEMONTH], \THEYEAR}

\author{AUTORNAME}
\title{Lösung AUFGABENTITEL}

\lstdefinelanguage{algopseudocode}
{
morekeywords={
class,if,fi,do,od,for,to,then,function,while,else,return,delete,new
},
morecomment=[l]{//}
}


\begin{document}
\maketitle
\lstset{language=algopseudocode}

\textbf{Aufgabe 10.1}

Beispiel:

\begin{minipage}[c]{\textwidth}
\begin{lstlisting}[frame=single,title={Aufgabe 10.3}]
int n <- Anzahl der Knoten;
Liste E1 <- Liste aller Kanten;
Liste E2 <- Liste aller Kanten;
Kante[] rev = new Kante[E1.length];

//Bucketsort(Liste, #Elemente, #Buckets, Sortierkriterium)

Bucketsort(E1, E1.length, n, v);
Bucketsort(E1, E1.length, n, u);

Bucketsort(E2, E2.length, n, u);
Bucketsort(E2, E2.length, n, v);

for int i=1 to E1.length do
	if (E1[i].u = E2[i].v) ^ (E1[i].v = E2[i].u) then
		rev[i] <- E2[i];
	else
		rev[i] <- null;
	fi
od
\end{lstlisting}
\end{minipage}

\vspace{0.75cm}

\end{document}
